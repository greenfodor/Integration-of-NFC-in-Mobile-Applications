\chapter{Conclusions}
\label{conclusions}

\par When talking about privacy and security, considering the improvements we have mentioned in chapter \ref{ch:6}, it is important that we learn the fact that no system can be completely secure and completely protected from prying eyes. A person which roots their device can easily bypass any flags that restricts screenshots and screen recordings as well as a biometric authentication in order to confirm their identity. However in those cases the biometric system can help making the data extraction harder giving enough time in some cases for the device to be either wiped or recovered.

However we will only be able to use the implementation we provided, as amazing as it is, in a restricted way. The reason behind our statement is that in order to take full advantage of it, a centralized system for all the health care institutes in a country would be necessary in order for all the necessary reports to be available when checking a patient's profile. We wish that this was not true, but centralizing all of the data in one big database presents a totally different security concern.

Keeping this in mind however, we still think there is potential for the NFC technology to be introduced in hospitals and medical institutes, even if in a much smaller scale, like a city for example. The benefits and the steps we could take forward using this approach in order to preserve the patient's privacy are worth the trouble.

To sum up everything that has been stated so far, we do believe that our solution, the "Medical Records" application, is an appropriate solution, and once improved with the features that we mentioned in chapter \ref{ch:6} could be modified to work with the existing EMR systems and used in order to meet the current standards regarding the privacy in the European Union.